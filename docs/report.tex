\documentclass[12pt,a4paper]{report}

\usepackage{alltt, fancyvrb, url}
\usepackage{graphicx}
\usepackage[utf8]{inputenc}
\usepackage{float}
\usepackage{xcolor}
\usepackage{hyperref}
\usepackage{longtable}
\usepackage{listings}
\usepackage{color}
\usepackage[utf8]{inputenc}

\usepackage{enumitem}
\usepackage{geometry}
\usepackage{pdfpages}

\geometry{margin=1in}

\usepackage{textcomp}
\usepackage{siunitx}

\usepackage[english]{babel}
\usepackage[capitalise, english]{cleveref}

\graphicspath{ {./src/img} }

\textwidth=450pt\oddsidemargin=0pt
\begin{document}

\begin{titlepage}
  \begin{center}
    {{\Large{\textsc{Alma Mater Studiorum $\cdot$ University of
    Bologna}}}}\\[1ex]
    \rule{15.8cm}{0.6mm}\\[2ex]
    {\small Bachelor's Degree in Computer Engineering and Science}\\
    {\small{ A.A. 2025/26}}
  \end{center}

  \vfill
  \begin{center}
    {\LARGE{\bf WEB25}}\\[1ex]
  \end{center}
  \vfill

  \begin{center}
    \begin{tabular}{@{}c@{\qquad}c@{}}
      {\large\bf Grazia Bochdanovits de Kavna} & {\large\bf
      Alessandro Rebosio} \\[0.5ex]
      {\small matr.\ 0001117082} & {\small matr.\ 0001130557}
    \end{tabular}
  \end{center}

  \vspace{2mm}
  \noindent\rule{\linewidth}{0.4pt}

\end{titlepage}

\tableofcontents

\chapter{Analisi}

\section{Descrizione}
UniMatch è un'applicazione web progettata per facilitare la connessione tra studenti universitari 
in cerca di compagni per progetti, studio di gruppo o preparazione agli esami.
Il progetto nasce dall’esigenza di trovare partner affidabili con interessi e competenze simili, 
cosa spesso difficile nelle università con numerosi corsi e studenti.
L’app adotta un approccio simile a un social network: gli studenti creano un account e possono 
proporre progetti indicando corso di studio, numero di collaboratori e competenze richieste. 
Gli altri utenti possono interagire con i post, e l’autore può contattare direttamente chi mostra interesse. 
In questo modo, UniMatch semplifica la ricerca di partner di studio compatibili 
e favorisce la collaborazione accademica.

\section{Analisi dei requisiti}
\subsection{Requisiti funzionali}

\subsubsection{Utente Studente}
\begin{itemize}
    \item \textbf{Registrazione e login}: possibilità di creare un account personale o accedere con credenziali già registrate.
    \item \textbf{Creazione e pubblicazione di post}: possibilità di proporre progetti indicando titolo, descrizione, corso di studio, numero di collaboratori richiesti e competenze necessarie.
    \item \textbf{Visualizzazione dei post}: possibilità di visualizzare i post pubblicati da altri utenti nella home.
    \item \textbf{Interazione con i post}: possibilità di mettere "mi piace" o "skip" ai post nella home.
    \item \textbf{Gestione profilo}: visualizzazione e modifica dei dati personali, elenco dei post pubblicati e lista degli utenti che hanno messo "mi piace" ai propri post.
    \item \textbf{Sistema di messaggistica e chat}: possibilità di inviare e ricevere messaggi privati con altri utenti interessati ai propri post.
\end{itemize}

\subsubsection{Utente Admin}
\begin{itemize}
    \item Accesso a tutte le funzionalità dell'utente studente.
    \item Monitoraggio dei post pubblicati da tutti gli utenti.
    \item Gestione degli account utenti (modifica o rimozione).
    \item Possibilità di gestire contenuti inappropriati o segnalati.
\end{itemize}

\subsection{Requisiti non funzionali}
\begin{itemize}
    \item \textbf{Usabilità}: l'interfaccia deve essere semplice, intuitiva e facilmente navigabile.
    \item \textbf{Accessibilità}: l'applicazione deve essere accessibile a utenti con disabilità, rispettando le linee guida WCAG.
    \item \textbf{Prestazioni}: l'applicazione deve garantire tempi di risposta rapidi durante la visualizzazione dei post e l'invio/ricezione dei messaggi.
    \item \textbf{Compatibilità}: l'applicazione deve essere fruibile su diversi browser web e dispositivi, inclusi desktop, smartphone e tablet.
    \item \textbf{Manutenibilità}: il codice deve essere chiaro e modulare per facilitare aggiornamenti e miglioramenti futuri.
\end{itemize}

\chapter{Scenari d'uso e Personas}

\chapter{Mockup dell'applicazione}

\chapter{Ambiente di sviluppo}

\appendix
\chapter{Guida Utente}

\end{document}
