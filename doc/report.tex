\documentclass[12pt,a4paper]{report}

\usepackage{alltt, fancyvrb, url}
\usepackage{graphicx}
\usepackage[utf8]{inputenc}
\usepackage{float}
\usepackage{xcolor}
\usepackage{hyperref}
\usepackage{longtable}
\usepackage{listings}
\usepackage{color}
\usepackage[utf8]{inputenc}

\usepackage{enumitem}
\usepackage{geometry}
\usepackage{pdfpages}

\geometry{margin=1in}

\usepackage{textcomp}
\usepackage{siunitx}

\usepackage[english]{babel}
\usepackage[capitalise, english]{cleveref}

\graphicspath{ {./src/img} }

\textwidth=450pt\oddsidemargin=0pt
\begin{document}

\begin{titlepage}
  \begin{center}
    {{\Large{\textsc{Alma Mater Studiorum $\cdot$ University of
    Bologna}}}}\\[1ex]
    \rule{15.8cm}{0.6mm}\\[2ex]
    {\small Bachelor's Degree in Computer Engineering and Science}\\
    {\small{ A.A. 2025/26}}
  \end{center}

  \vfill
  \begin{center}
    {\LARGE{\bf WEB25}}\\[1ex]
  \end{center}
  \vfill

  \begin{center}
    \begin{tabular}{@{}c@{\qquad}c@{}}
      {\large\bf Grazia Bochdanovits de Kavna} & {\large\bf
      Alessandro Rebosio} \\[0.5ex]
      {\small matr.\ 0001117082} & {\small matr.\ 0001130557}
    \end{tabular}
  \end{center}

  \vspace{2mm}
  \noindent\rule{\linewidth}{0.4pt}

\end{titlepage}

\tableofcontents

\chapter{Analisi}
\section{Descrizione}
UniEvents è una web application per la creazione, la gestione e la
scoperta di eventi universitari (workshop, conferenze, seminari e
attività sociali). Fornisce agli organizzatori strumenti per
impostare eventi con metadati (titolo, descrizione, data/ora, luogo,
immagine, capienza) e ai partecipanti un'interfaccia per cercare,
filtrare e iscriversi agli eventi.

\subsection*{Requisiti Funzionali}
\begin{itemize}
  \item Creazione e modifica di eventi con campi: titolo,
    descrizione, data/ora, luogo, immagine e numero massimo partecipanti.
  \item Anteprima live durante la creazione (immagine e metadati
    aggiornati in tempo reale).
  \item Possibilità di salvare bozze e pubblicare eventi quando pronti.
  \item Registrazione degli utenti agli eventi e conteggio
    partecipanti con limite massimo.
  \item Ricerca e filtri per categoria e data nella pagina principale.
\end{itemize}

\subsection*{Requisiti Non Funzionali}
\begin{itemize}
  \item Accessibilità: i componenti interattivi (modali, form) devono
    rispettare le linee guida ARIA e la gestione del focus.
  \item Usabilità: interfaccia semplice e reattiva; \newline anteprime
    immediate per feedback all'utente.
  \item Performance: caricamento veloce delle liste e gestione
    efficiente delle immagini (lazy-loading quando possibile).
  \item Portabilità e responsive design: funzionamento corretto su
    dispositivi mobile e desktop.
\end{itemize}

\chapter{Personas e Scenarios}
\section*{Personas}
\begin{itemize}
  \item \textbf{Martina (Studentessa, 21 anni)}: cerca workshop e
    seminari tramite smartphone. Vuole trovare eventi rapidamente e
    iscriversi con pochi passaggi.
  \item \textbf{Prof. Rossi (Organizzatore)}: pubblica conferenze e
    controlla le iscrizioni da desktop; usa bozze per preparare
    contenuti prima della pubblicazione.
  \item \textbf{Lucia (Admin)}: gestisce categorie, rimuove contenuti
    inappropriati e controlla statistiche di utilizzo.
\end{itemize}

\section*{Scenarios}
\begin{enumerate}
  \item \textbf{Creare e pubblicare un evento (Organizzatore)}:
    l'organizzatore apre la modale di creazione, compila i campi,
    trascina un'immagine nella dropzone, verifica l'anteprima e
    pubblica. L'evento appare nella homepage.
  \item \textbf{Iscriversi a un evento (Studente)}: lo studente
    filtra per categoria, apre il dettaglio e clicca "Iscriviti";
    riceve conferma e il contatore partecipanti si aggiorna.
  \item \textbf{Moderazione (Admin)}: l'admin individua un evento
    segnalato, apre il pannello di moderazione e decide di rimuoverlo
    o segnalarlo come non conforme.
\end{enumerate}

\chapter{Mockup}
\begin{figure}[htbp]
  \centering
  \includegraphics[width=\textwidth]{img/index.png}
  \caption{Mockup della homepage di UniEvents}
  \label{fig:mockup-homepage}
\end{figure}

\appendix
\chapter{Guida Utente}

\end{document}
