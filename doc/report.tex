\documentclass[12pt,a4paper]{report}

\usepackage{alltt, fancyvrb, url}
\usepackage{graphicx}
\usepackage[utf8]{inputenc}
\usepackage{float}
\usepackage{xcolor}
\usepackage{hyperref}
\usepackage{longtable}
\usepackage{listings}
\usepackage{color}
\usepackage[utf8]{inputenc}

\usepackage{enumitem}
\usepackage{geometry}
\usepackage{pdfpages}

\geometry{margin=1in}

\usepackage{textcomp}
\usepackage{siunitx}

\usepackage[english]{babel}
\usepackage[capitalise, english]{cleveref}

\graphicspath{ {./src/img} }

\textwidth=450pt\oddsidemargin=0pt
\begin{document}

\begin{titlepage}
  \begin{center}
    {{\Large{\textsc{Alma Mater Studiorum $\cdot$ University of
    Bologna}}}}\\[1ex]
    \rule{15.8cm}{0.6mm}\\[2ex]
    {\small Bachelor's Degree in Computer Engineering and Science}\\
    {\small{ A.A. 2025/26}}
  \end{center}

  \vfill
  \begin{center}
    {\LARGE{\bf WEB25}}\\[1ex]
  \end{center}
  \vfill

  \begin{center}
    \begin{tabular}{@{}c@{\qquad}c@{}}
      {\large\bf Grazia Bochdanovits de Kavna} & {\large\bf
      Alessandro Rebosio} \\[0.5ex]
      {\small matr.\ 0001117082} & {\small matr.\ 0001130557}
    \end{tabular}
  \end{center}

  \vspace{2mm}
  \noindent\rule{\linewidth}{0.4pt}

\end{titlepage}

\tableofcontents

\chapter{Analisi}
\section{Descrizione}
UniEvents è una web application per la creazione, la gestione e la
scoperta di eventi universitari (workshop, conferenze, seminari e
attività sociali). Fornisce agli organizzatori strumenti per
impostare eventi con metadati (titolo, descrizione, data/ora, luogo,
immagine, capienza) e ai partecipanti un'interfaccia per cercare,
filtrare e iscriversi agli eventi.

\subsection*{Requisiti Funzionali}
\begin{itemize} 
\item \textit{Gestione Eventi:} Creazione e modifica di eventi con campi obbligatori 
(titolo, descrizione, data, ora, luogo, categoria, capienza partecipanti e una immagine).
\item \textit{Stati dell'Evento:} Possibilità di salvare gli eventi come bozze (Draft), 
inviarli in revisione (Waiting) o pubblicarli. 
\item \textit{Anteprima Dinamica:} Visualizzazione di un'anteprima della card dell'evento durante 
la fase di creazione, con aggiornamento istantaneo di testi e categorie. 
\item \textit{Sistema di Iscrizione:} Registrazione degli utenti agli eventi con controllo 
automatico della disponibilità dei posti. 
\item \textit{Ricerca Avanzata e Filtri:} Filtraggio degli eventi per categoria e per ricerca 
testuale.
\item \textit{Dashboard Storico:} Visualizzazione separata degli eventi passati e di quelli 
annullati per monitorare l'attività pregressa dell'utente.
\end{itemize}

\subsection*{Requisiti Non Funzionali}
\begin{itemize}
  \item \textit{Accessibilità:} i componenti interattivi (modali, form) devono
    rispettare le linee guida ARIA e la gestione del focus.
  \item \textit{Usabilità:} interfaccia semplice e reattiva;  anteprime
    immediate \newline per feedback all'utente.
  \item \textit{Portabilità e responsive design:} funzionamento corretto su
    dispositivi mobile e desktop.
\end{itemize}

\chapter{Utenti e scenari d'uso}
\section{Personas}
\begin{itemize}
  \item \textbf{Martina (Studentessa, 21 anni)}: cerca workshop e
    seminari tramite smartphone. Vuole trovare eventi rapidamente e
    iscriversi con pochi passaggi.
  \item \textbf{Prof. Rossi (Organizzatore)}: pubblica conferenze e
    controlla le iscrizioni da desktop; usa bozze per preparare
    contenuti prima della pubblicazione.
  \item \textbf{Lucia (Admin)}: gestisce categorie, rimuove contenuti
    inappropriati e controlla statistiche di utilizzo.
\end{itemize}

\section{Scenarios}
\begin{enumerate}
  \item \textbf{Creare e pubblicare un evento (Organizzatore)}:
    l'organizzatore apre la modale di creazione, compila i campi,
    trascina un'immagine nella dropzone, verifica l'anteprima e
    pubblica. L'evento appare nella homepage degli utenti che possono iscriversi.
  \item \textbf{Iscriversi a un evento (Studente)}: lo studente
    filtra gli eventi in base alle sue necessità e clicca "Iscriviti";
    riceve conferma e il contatore partecipanti si aggiorna.
  \item \textbf{Moderazione (Admin)}: l'admin approva eventi in attesa di essere resi pubblici,
    oppure decide di rimuoverli perchè non conformi alle linee guida della comunity.
\end{enumerate}

\appendix
\chapter{Guida Utente}

\section{Creazione del database}
Per creare il database MySQL a partire dagli script SQL forniti,
assicurarsi di avere MySQL installato e in esecuzione.

\begin{verbatim}
> mysql < resources/database.sql
> mysql < resources/demo.sql
\end{verbatim}

\section{Avvio dell'applicazione}
Il progetto è basato su PHP. Per impostare le credenziali per la
connessione al database modificare il file \texttt{app/bootstrap.css}
(o il file di configurazione equivalente) e inserire host, nome
database, utente e password.

\vspace{\baselineskip}
\noindent Per avviare l'applicazione usare il server integrato di PHP:

\begin{verbatim}
> php -S localhost:8080 -t public
\end{verbatim}

L'applicazione sarà accessibile all'indirizzo
\url{http://localhost:8080/}. Effettuare il login o la registrazione
per iniziare a utilizzare il sistema.

\section{Credenziali demo per accesso}
Segue una tabella con gli account presenti nello script di
popolazione \newline \texttt{resources/demo.sql}.

\begin{center}
  \begin{tabular}{ c c c c }
    \hline
    Email & Ruolo & Password \\
    \hline
    mario.rossi@example.com & USER & pas123 \\
    luisa.bianchi@example.com & USER & pas123 \\
    giulia.verdi@example.com & HOST & pas123 \\
    paolo.neri@example.com & HOST & pas123 \\
    admin@web25.com & ADMIN & pas123 \\
    \hline
  \end{tabular}
\end{center}

\paragraph{Nota:} Le password nello script
\texttt{resources/demo.sql} sono memorizzate come hash.

\end{document}
